\section{Prefazione} \label{sec:prefazione}
    Con questo documento QB Software intende normare i propri processi per lo sviluppo di un progetto software, tali processi sono stati creati a partire dallo standard proposto \cite{bib:ISO12207_1997}.

    Questo documento si struttura in modo simile allo standard \cite{bib:ISO12207_1997}, ogni processo è indicato da un numero (a)%
    \footnote{Eccezione per la sezione: \ref{sec:prefazione} prefazione.}%
    , ogni attività è indicata dal numero (a.b), e ogni task è indicata dalla numerazione (a.b.c), in fine introduciamo anche una numerazione (a.b.c.d) per fornire un ulteriore granularità con le regole e i passi da seguire in una procedura. La scelta di avere un documento per struttura allo standard ci permette di mantenere il più fedelmente possibile le linee guida dettate dallo standard ISO. Le norme presentate in questo documento hanno lo scopo di essere il più prescrittive possibile, al fine di definire nel modo più "algoritmico" possibile le procedure di lavoro, e limitare fortemente lo spazio a scelte di libero arbitrio che rischiano di portare a situazioni meno controllate rispetto a quelle previste durante la stesura del way of working. 

    Prima di leggere un qualunque documento prodotto da QB Software è obbligatorio leggere il glossario presente nel \href{https://github.com/QB-Software-swe/docs}{repository GitHub con la documentazione di QB Software}.

    \begin{center}
        \begin{tabularx}{0.85\textwidth}{>{\centering\arraybackslash}X}
            \toprule
            Ogni membro del gruppo si impegna a leggere, a comprendere, e a mettere in pratica in pieno le norme presenti in questo documento.
            \\\bottomrule
        \end{tabularx}
    \end{center}