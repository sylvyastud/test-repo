\subsection{Processo di documentazione} \label{sec:documentation}
    Il processo di documentazione ha lo scopo di registrate tutte le informazioni prodotte durante il ciclo di vita di un processo.

    \subsubsection{Implementazione del processo}
        In questa attività normiamo la pianificazione dello sviluppo dei documenti, riportiamo tutta la documentazione da produrre durante il ciclo di vita del prodotto software.
        
        \paragraph{Norme di progetto}
            \begin{enumerate}
                \item \textbf{Scopo}, normare il way of working in modo prescrittivo, regolando i vari processi proposti dello standard \cite{bib:ISO12207_1997}, basandosi sulle decisioni prese durante le riunioni;
                \item \textbf{target}, team di QB Software, documento interno;
                \item \textbf{procedure e responsabilità},
                    \newline
                    \begin{tabularx}{0.93\textwidth}{|X|X|X|}
                        \hline
                        \textbf{Procedura} & \textbf{responsabilità} & \textbf{Sezione di} \\
                        & & \textbf{riferimento (NdP)} \\
                        \hline
                        Sviluppo & Amministratore &  \ref{sec:doc_life_cycle}
                        \\\hline
                        Manuntenzione & Amministratore & \ref{sec:doc_maintenance} 
                        \\\hline
                        Verifica & Verificatore & \ref{sec:doc_verification}
                        \\\hline
                    \end{tabularx}
                \item \textbf{schedule}, il documento non prevede una versione finale in quanto è un documento incrementale, cioè viene aggiornato quando è necessario, la prima versione è prevista per il 13/11/2023. 
            \end{enumerate} 

        \paragraph{Verbali interni}
            \begin{enumerate}
                \item \textbf{Scopo}, riportare le decisioni prese durante le riunioni interne ufficiali. Il documento ha come primo obbiettivo quello di ripotare le decisioni di pianificazione prese, con annesse le ragioni della decisioni, e i ticket inseriti nel ITS dovuti ai compiti assegnati, e gli argomenti da trattare per la prossima riunione;
                \item \textbf{target}, team di QB Software, documento interno;
                \item \textbf{procedure e responsabilità},
                    \newline
                    \begin{tabularx}{0.93\textwidth}{|X|X|X|}
                        \hline
                        \textbf{Procedura} & \textbf{responsabilità} & \textbf{Sezione di} \\
                        & & \textbf{riferimento (NdP)} \\
                        \hline
                        Sviluppo & ???? &  \ref{sec:doc_life_cycle}
                        \\\hline
                        Manuntenzione & - & -
                        \\\hline
                        Verifica & Verificatore & \ref{sec:doc_verification}
                        \\\hline
                    \end{tabularx}
                \item \textbf{schedule}, il verbale deve essere redatto e verificato entro 2 giorni dall'avvenuta riunione.
            \end{enumerate} 

        \paragraph{Verbali esterni}
            \begin{enumerate}
                \item \textbf{Scopo}, riportare le decisioni prese durante le riunioni esterni ufficiali. Il documento come primo obbiettivo quello di ripotare gli argomenti di discussione durante la riunione, e i ticket inseriti nel ITS dovuti alle decisioni;
                \item \textbf{target}, QB Software, documento esterno;
                \item \textbf{Procedure e responsabilità},
                    \newline
                    \begin{tabularx}{0.93\textwidth}{|X|X|X|}
                        \hline
                        \textbf{Procedura} & \textbf{responsabilità} & \textbf{Sezione di} \\
                        & & \textbf{riferimento (NdP)} \\
                        \hline
                        Sviluppo & ??? &  \ref{sec:doc_life_cycle}
                        \\\hline
                        Manuntenzione & - & -
                        \\\hline
                        Verifica & Verificatore & \ref{sec:doc_verification}
                        \\\hline
                        Approvazione & Esterni & ????
                        \\\hline
                    \end{tabularx}
                \item \textbf{Schedule}: il verbale deve essere redatto e verificato entro 2 giorni dall'avvenuta riunione internamente da QB Software, poi l'approvazione passa a tutti i proponenti esterni.
            \end{enumerate} 

        \paragraph{Piano di Qualifica}
        \begin{enumerate}
            \item \textbf{Scopo}, normare le procedure di verifica;
            \item \textbf{target}, QB Software, documento esterno;
            \item \textbf{Procedure e responsabilità},
                \newline
                \begin{tabularx}{0.93\textwidth}{|X|X|X|}
                    \hline
                    \textbf{Procedura} & \textbf{responsabilità} & \textbf{Sezione di} \\
                    & & \textbf{riferimento (NdP)} \\
                    \hline
                    Sviluppo & ??? &  \ref{sec:doc_life_cycle}
                    \\\hline
                    Manuntenzione & - & -
                    \\\hline
                    Verifica & Verificatore & \ref{sec:doc_verification}
                    \\\hline
                    Approvazione & Esterni & ????
                    \\\hline
                \end{tabularx}
            \item \textbf{Schedule}: il verbale deve essere redatto e verificato entro 2 giorni dall'avvenuta riunione internamente da QB Software, poi l'approvazione passa a tutti i proponenti esterni.
        \end{enumerate} 

        \paragraph{Lettera di presentazione (Candidatura)}
            \begin{enumerate}
                \item \textbf{Scopo}: presentarsi alla candidatura, riportando i documenti e brevemente le decisioni più rillevanti;
                \item \textbf{Target}: committente;
                \item \textbf{Procedura e responsabilità}: nessuna procedura è stata attuata durante questo periodo;
                \item \textbf{schedule}: deadline 31/10/2023, ore 17:00.
            \end{enumerate} 

        \paragraph{Preventivo costi e degli impegni (Candidatura)}
            bla bla bla

        \paragraph{Valutazione capitolati (Candidatura)}
            bla bla bla

        \paragraph{Analisi dei requisiti (RTB)}
            bla bla bla

        % Struttura ipotetica, da revisionare

    \subsubsection{Design e development}
        In questa attività vengono riportati: tutti gli strumenti necessari allo sviluppo della documentazione, e come devono essere imapginati i documenti prodotti. 
    
        \paragraph{Strumenti per la stesura dei documenti}
            \begin{itemize}
                \item I documenti devono essere scritti in \LaTeX, usando la distribuzione \href{https://tug.org/texlive/}{TeX Live};
                \item ogni documento deve importare il pacchetto \LaTeX\ \texttt{qbsoftware.sty}, il quale contiene tutte le utilità e le regole tipografiche normate in questo documento per lo sviluppo della documentazione;
                \item vengono messi a disposizione i seguenti template, presenti nella cartella docs\_src/templates:
                \begin{itemize}
                    \item la cartella \verb|empy|, struttura di un documento di base generico, da questo template derivano tutti gli altri template;
                    \item la cartella \verb|verbale_interno|, struttura di un documento per i verbali interni;
                    \item la cartella \verb|verbale_esterno|, struttura di un documento per i verbali esterni.
                \end{itemize}
                \item GitHub tracciamento storia sorgenti dei documenti nel ramo \verb|develop|;
                \item GitHub tracciamento PDF dei documenti verificati e approvati nel ramo \verb|main|.
            \end{itemize}
    
        \paragraph{Impaginazione di base}
            Ogni documento di QB Software deve essere sviluppato a partire da un template di base, presente nella cartella docs\_src/ templates/empty. Il template di base deve rispettare la seguente impaginazione:
            \begin{enumerate}[label=\Roman*)]
                \item deve essere in formato A4, dimensione font 12pt;
                \item la prima pagina deve riportare nel seguente ordine:
                \begin{enumerate}[label=\arabic*.]
                    \item la scritta "QB Software";
                    \item il logo di QB Software;
                    \item il logo dell'università di Padova;
                    \item la scritta "\textsc{Università degli studi di Padova}";
                    \item la scritta "\textsc{corso di ingegneria del software}";
                    \item la scritta "\textsc{anno accademico 2023/2024}";
                    \item il titolo del documento, e quando richiesto anche la data;
                    \item il contatto e-mail di QB Software.
                \end{enumerate}
                \item la seconda pagina è dedicata al registro delle modifiche descritto al paragrafo \ref{sec:doc_changelog};
                \item una pagina dedicata all'indice dei contenuti generato da \LaTeX;
            \end{enumerate}
            %
            Ogni documento deve riportare su ogni pagina, a eccezione della prima pagina, un piè di pagina e un testatina separate dalla gabbia con una linea. In ogni testatina deve essere riportato nel margine destro il logo del gruppo e nel margine sinistro la scritta "QB Software". Ogni piè di pagina deve riportare nel margine sinistro il titolo del documento e nel margine destro la pagina attuale nella seguente forma \verb|Pagina x di y|, dove \verb|x| è la pagina attuale, e \verb|y| il totale delle pagine senza contare la prima.
        
        \paragraph{Regole tipografiche}
            Di seguito ridefiniamo, o aggiungiamo, ulteriore regole tipografiche oltre a quelle normalmente usate dal \LaTeX, con lo scopo di rendere il documento più accessibile, ed evitare incongruenze di stile tra i documenti:
            \begin{itemize}
                \item ogni tabella e figura presenti nel documento, a eccezione del \emph{registro delle modifiche} e dei loghi, devono essere accompagnati da una didascalia che ne descrive il contenuto, a questo scopo usare l'ambiente LaTeX \verb|figure| o \verb|table| e l'istruzione \verb|\caption| per la didascalia;
                \item ogni tabella e figura, inoltre devono avere una label che viene creata con il commando \verb|\label|. Le label devono iniziare come \emph{fig:} per le figure, e \emph{table:} per le tabelle;
                \item le tabelle vanno inserite in un ambiente \verb|table| e devono essere posizionate sempre all'inizio della pagina, come da impostazione predefinita per l'ambiente citato;
                \item quando ci si riferisce ad una figura, o a una tabella, o a una sezione, citarla con il comando \verb|\ref| specificando la tipologia (tabella, figura, sezione) dell'elemento citato seguito dal suo numero identificativo
                \item ogni link deve essere inserito sotto forma di testo attraverso sottolineato di colore blu, inoltre non si deve scrivere direttamente l'URL, ma una frase chiara che specifichi dove quel link stia puntando;
                \item solo, e soltanto i link posso essere sottolineati, nessun'altra parte del testo può essere sottolineata;
                \item ogni sezione creata con il commando \verb|\section| deve iniziare sempre in una nuova pagina, per fare ciò ogni section di testo va scritta in un file .tex a parte, sotto la cartella \verb|sections/| e importanto nel documento principale attraverso il comando \verb|\include|;
                \item --
            \end{itemize}

        \paragraph{Registro delle modifiche} \label{sec:doc_changelog}
            Il registro delle modifiche, per tendere una traccia completa e sensata della storia del documento deve riportare i seguenti dati:
            \begin{itemize}
                \item versione del documento;
                \item data della modifica;
                \item autore della modifica;
                \item ruolo assunto dall'autore al momento della stesura;
                \item chi si è occupato della verifica;
                \item data del superamento della verifica;
                \item descrizione, breve, ma significativa delle modifiche apportante, con riferimento alla sezione modificata.
            \end{itemize}
            %
            Il registro delle modifiche deve essere implementato attraverso l'ambiente \LaTeX\ \verb|changelog| definito all'interno del pacchetto \LaTeX: \verb|qbsoftware.sty|, tale ambiente deve provvedere a creare:
            \begin{enumerate}
                \item il titolo "Registro delle modifiche", il quale non verrà riportato nell'indice del documento;
                \item una tabella formata dalle seguenti quattro colonne, nel seguente ordine:
                \begin{enumerate}
                    \item \emph{V.}, vengono riportate le versioni del documento al momento dell'approvazione della modifica;
                    \item \emph{Data}, vengono riportate le date di stesura della modifica e di approvazione da parte del verificatore;
                    \item \emph{Autore}, vengono riportati gli autori della modifica, e i verificatori che hanno approvato la modifica;
                    \item \emph{Ruolo}, vengono riportati i ruoli degli autori al momento della modifica, per chi ha fatto la verifica viene riportato il ruolo di verificatore;
                    \item \emph{Descrizione}, vengono riportate le modifiche, o aggiunte, fatte al documento facendo riferimento alle sezioni che hanno subito la modifica, o aggiunta. 
                \end{enumerate}
            \end{enumerate}
            %
            Inoltre nel pacchetto \verb|qbsoftware.sty| viene fornito il comando:
            \begin{center}
                \verb*|\newlog{Ver}{Data}{Autore}{RuoloAutore}{Verificatore}{Desc}|
            \end{center}
            che permette di inserire una nuova modifica all'interno del registro delle modifiche. Il comando deve essere usato all'interno dell'ambiente \verb|changelog| dentro il pacchetto citato prima.

        \paragraph{Versionamento} \label{sec:doc_versionamento}
            La versione dei documenti proposta deriva dal \href{https://semver.org/}{semantic versioning} ed è composta da 2 cifre:
            \begin{center}
                $x.y$
            \end{center}
            \begin{itemize}
                \item $x$ rappresenta una modifica sonstanziale, come un'aggiunta di una nuova sezione, [Definire meglio];
                \item $y$ rappresenta una modifica minore, come l'aggiornamento di una paragrafo, o la modifica dell'impaginazione [Definire meglio];
            \end{itemize}
            Non è stata usata una terza cifra, in quanto il gruppo considera inutile indicare [Boh, forse togliere questa frase?].


        \paragraph{Ciclo di vita dei documenti} \label{sec:doc_life_cycle}
            Di seguito illustriamo delle fasi del ciclo di vita di ogni documento, ogni creazione/modifica di un documento deve essere collegato ad una issues,
            [MANCA REG MOD. E UN IDEA MIGLIORE DI COME FARE LE COSE]
            \begin{enumerate}
                \item su GitHub viene creato un feature branch con GitFlow, con il nome \verb|new_<nome_documento>|;
                \item l'incaricato redige il documento, ogni volta che termina il valoro deve fare un push del lavoro svolto sul branch, così tutto il gruppo può vedere lo stato dei lavori, e contribuire se l'attività era pianificata per più persone;
                \item quando i readattori considerano le loro modifiche approvate devono fare una richiesta di pull nel ramo di develop;
                \item il verificatore incaricato si segna come verificatore della pull request, e procedere a verificare il documento seguendo la checklist proposta nel Piano di Qualifica, che verifica la conformità del documento con quanto indicato nelle norme di progetto, sezione \ref{sec:documentation} e le richieste esposte nella issues;
                \item il verificatore una volta completato la checklista, approva solo la pull request quando la checklist è completamente superata, la checklist va riportata su un commento in GitHub della pull request indicando se è stata passata o no;
                \item nel caso di checklist, gli incaricati della stesura procedono a rendere confrome il documento, tornando al punto 2;
                \item nel caso di approvazione, viene fatto il merge in develop, e deve essere eliminato il branch di feature in quanto è stato commitato in develop.
                \item il documento viene compilato a parte, in formato PDF, e viene caricato nel main secondo la disposizione indicata dal proprio CI che si trova nel file \verb|cm.csv|, per vedere il funzionamento leggere il processo di CM \ref{sec:cm};
            \end{enumerate}
            Per la manuntenzione del documento vedere la sezione \ref{sec:doc_maintenance}
    \subsubsection{Produzione}
        \paragraph{Verifica del documento} \label{sec:doc_verification}

        \paragraph{Mettere in produzione il documento}


        \paragraph{Configuration Managment per i documenti}

    \subsubsection{Manutenzione} \label{sec:doc_maintenance}